\documentclass[12pt,letterpaper]{article}
\usepackage{fullpage}
\usepackage[top=2cm, bottom=4.5cm, left=2.5cm, right=2.5cm]{geometry}
\usepackage{amsmath,amsthm,amsfonts,amssymb,amscd}
\usepackage{lastpage}
\usepackage{enumerate}
\usepackage{fancyhdr}
\usepackage{mathrsfs}
\usepackage{xcolor}
\usepackage{graphicx}
\usepackage{listings}
\usepackage{hyperref}

\hypersetup{%
  colorlinks=true,
  linkcolor=blue,
  linkbordercolor={0 0 1}
}
 
\renewcommand\lstlistingname{Algorithm}
\renewcommand\lstlistlistingname{Algorithms}
\def\lstlistingautorefname{Alg.}

\lstdefinestyle{Python}{
    language        = Python,
    frame           = lines, 
    basicstyle      = \footnotesize,
    keywordstyle    = \color{blue},
    stringstyle     = \color{green},
    commentstyle    = \color{red}\ttfamily
}

\setlength{\parindent}{0.0in}
\setlength{\parskip}{0.05in}

% Edit these as appropriate
\newcommand\course{MCS 268P}
\newcommand\hwnumber{1}                  % <-- homework number
\newcommand\NetIDa{YOUR NAME HERE}           % <-- NetID of person #1
\newcommand\NetIDb{HWK 1 Part 1}           % <-- NetID of person #2 (Comment this line out for problem sets)

\pagestyle{fancyplain}
\headheight 35pt
\lhead{\NetIDa}
\lhead{\NetIDa\\\NetIDb}                 % <-- Comment this line out for problem sets (make sure you are person #1)
\chead{\textbf{\Large Formulating LPs and ILPs}} %\hwnumber}}
\rhead{\course \\ \today}
\lfoot{}
\cfoot{}
\rfoot{\small\thepage}
\headsep 1.5em

\begin{document}

\section*{Notes}
Using the skills you  from the Gurobi Tutorials on Linear Programming, formulate the following simple problems and solve them graphically. Check your answers using Gurobi or if you know how, in Excel. 

\section*{Problem 1}

\begin{itemize}
    \item A merchant  plans  to  sell  two  models  of high end laptops  at 
costs of \$2500 and \$3000, respectively.
\item The \$2500 model yields a
profit  of  \$450  and  the  \$3000  model  yields  a  profit  of  \$550. 
\item The merchant  estimates  that  the  total  monthly  demand will not exceed 250 units. Further, the merchant has promised a university at least 100 of the less expensive models. 
\item Formulate the profit maximization model as a linear program. 
\item Assume  that
the  merchant  does  not  want  to  invest  more  than  \$650,000  in
computer inventory. To simplify -- just assume that his investment is equal to the sales price of each computer. 
\item Find the Optional Solution -- The number of units of each model that should  be  stocked  in  order  to  maximize  profit.  

\end{itemize}   

\newpage
\section*{Problem 2}

\begin{itemize}

\item A fruit  grower  has  150  acres  of  land  available  to  raise two crops, A and B. 
\item It takes one day to trim an acre of crop A and two days to trim an acre of crop B, and there are 240 days per year available for trimming. 
\item It takes 0.3 day to pick an acre of
crop A and 0.1 day to pick an acre of crop B, and there are 30 days  per  year  available  for  picking.  
\item Formulate this Problem as a Linear Program 
\item Optional -- Find  the  number  of  acres of each fruit that should be planted to maximize profit,  assuming  that  the  profit is \$140  per  acre  for  crop A and  \$235
per acre for crop B.

\end{itemize}

\newpage
\section*{Problem 3}


\begin{itemize}
    \item A farming  cooperative  mixes  two  brands  of  cattle  feed.  
    \item  Brand
X  costs  \$25  per  bag  and  contains  2  units  of  nutritional  element  A,  2  units  of  element  B,  and  2  units  of  element  C.
\item Brand  Y costs  \$20  per  bag  and  contains  1  unit  of  nutritional
element  A,  9  units  of  element  B,  and  3  units  of  element  C.
\item Formulate this problem as a linear program 
\item Now find  the  number  of  bags  of  each  brand  that  should  be  mixed to  produce  a  mixture  having  a  minimum  cost  per  bag.  The minimum  requirements  of  nutrients A,  B,  and  C  are  12  units, 36 units, and 24 units, respectively. 
\end{itemize}

\end {document}

\begin{enumerate}
  \item
   Problem 1 part 1 answer here.
  \item
    Problem 1 part 2 answer here.

    Here is an example typesetting mathematics in \LaTeX
\begin{equation*}
    X(m,n) = \left\{\begin{array}{lr}
        x(n), & \text{for } 0\leq n\leq 1\\
        \frac{x(n-1)}{2}, & \text{for } 0\leq n\leq 1\\
        \log_2 \left\lceil n \right\rceil \qquad & \text{for } 0\leq n\leq 1
        \end{array}\right\} = xy
\end{equation*}

    \item Problem 1 part 3 answer here.

    Here is an example of how you can typeset algorithms.
    There are many packages to do this in \LaTeX.
     
    \lstset{caption={Caption for code}}
    \lstset{label={lst:alg1}}
     \begin{lstlisting}[style = Python]
    from package import Class # Mesh required for..
    
    cinstance = Class.from_obj('class.obj')
    cinstance.go()
    \end{lstlisting}
     
  \item Problem 1 part 4 answer here.

    Here is an example of how you can insert a figure.
    \begin{figure}[!h]
    \centering
    \includegraphics[width=0.3\linewidth]{heidi.jpg}
    \caption{Heidi attacked by a string.}
    \end{figure}
\end{enumerate}


\section*{Problem 2}
% Rest of the work...

\end{document}
